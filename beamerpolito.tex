\documentclass{beamer}
\usepackage[utf8]{inputenc}
\usepackage{amsfonts,amsmath,oldgerm}
\usetheme{sintef}

\usefonttheme[onlymath]{serif}

\titlebackground*{assets/background}

\newcommand{\hrefcol}[2]{\textcolor{cyan}{\href{#1}{#2}}}

\title{Riconoscimento di dispositivi di
protezione individuale in ambito
industriale tramite infrastruttura
cloud}
%\subtitle{Sottotitolo della Tesi (se necessario)}
\course{Corso di Laurea Magistrale in Ingegneria Informatica}
\author{\href{mailto:rei.zoto@studenti.polito.it}{Rei Zoto}}
\IDnumber{Matricola: 258017}

\begin{document}
\maketitle

\section{Introduzione}

\begin{frame}{Problemi e soluzioni}
\begin{itemize}
    \item Panoramica infortuni sul lavoro.
    	\begin{itemize}
    		\item costi diretti
    		\item costi indiretti
    		\item impatto sulla società
    	\end{itemize}	
    \item dispositivi di sicurezza
    \item sistemi automatici al posto di controlli manuali
\end{itemize}
\end{frame}

\begin{frame}{Statistiche}
\begin{itemize}
    \item INAIL
    \item OSHA-EU
\end{itemize}

\end{frame}

\begin{frame}{Obiettivi e Motivazioni}
\textbf{Scopo del lavoro}:
\begin{itemize}
    \item Ricerca sull'utilizzo di modelli pronti all'uso forniti da servizi cloud.
    \item Implementazione di un sistema basato sul cloud per la rilevazione di dispositivi di sicurezza.
\end{itemize}

\textbf{Motivazioni personali}:
\begin{itemize}
    \item Interesse nelle tematiche IoT.
    \item Conoscenze pregresse di Machine Learning dal corso seguito al Politecnico.
    \item Scelta della tesi durante lo studio di sistemi operativi, virtualizzazione ed estensione dei concetti al cloud.
\end{itemize}
\end{frame}







\section{Panoramica}
\begin{frame}{Computer Vision}
\end{frame}
\begin{frame}{Lavori Correlati}
\begin{itemize}
    \item Approccio edge-cloud in collaborazione con Microsoft.
    \item Approccio on-premise (Stato dell'Arte).
    \item Citazione delle università coinvolte.
\end{itemize}

\textbf{Posizionamento della soluzione proposta}:
\begin{itemize}
    \item Dove si posiziona la soluzione rispetto ai lavori esistenti.
    \item Modello utilizzato: Rekognition
\end{itemize}
\end{frame}

\begin{frame}{Approccio Proposto}
\textbf{Use Case}:
\begin{itemize}
    \item Risposta al problema con un sistema near real-time.
\end{itemize}

\textbf{Motivazioni}:
\begin{itemize}
    \item Mancanza di benchmark specifici per dispositivi di sicurezza.
    \item L'approccio near real-time è conservativo a causa di:
    \begin{itemize}
        \item Tempi di risposta del modello non veloci (servizio pensato per tutti gli utenti AWS, tipicamente 5 fps).
        \item Latenza intrinseca del cloud computing e problemi di connettività.
    \end{itemize}
    \item Come risolvere il problema andando offline?
\end{itemize}
\end{frame}

\begin{frame}{Tecnologie Utilizzate}
    \centering
    \begin{tabular}{ccc}
        % Prima Riga: Docker, AWS, Apache Flink
        \includegraphics[width=0.2\textwidth]{images/docker_logo.png} & 
        \includegraphics[width=0.2\textwidth]{images/aws_logo.png} & 
        \includegraphics[width=0.2\textwidth]{images/apache_flink_logo.jpeg} \\
        Docker & AWS & Apache Flink \\
        \vspace{0.5cm} \\ % Spazio aggiuntivo tra le righe
        % Seconda Riga: MQTT, RTSP, Publish/Subscribe
        \includegraphics[width=0.1\textwidth]{images/mqtt_logo.png} & 
        \includegraphics[width=0.1\textwidth]{images/rtsp_logo.png} \\ 
        MQTT & RTSP \\
    \end{tabular}
    \footlinecolor{sintefyellow}
\end{frame}


\section{Architettura del Sistema}

\begin{frame}{Architettura del Sistema}
%\begin{figure}
%    \centering
%    \includegraphics[width=\textwidth]{assets/architettura_sistema}
%    \caption{Diagramma dell'Architettura Generale}
%\end{figure}
\begin{itemize}
    \item \textbf{Componenti Chiave}:
    \begin{itemize}
        \item Dispositivi Edge.
        \item Cloud.
        \item Pipeline di dati.
    \end{itemize}
\end{itemize}
\end{frame}

\section{Risultati}

\begin{frame}{Test Case}
\textbf{Funzionalità Raggiunte}:
\begin{itemize}
    \item Rilevazioni near real-time nei 5 use case.
    \item Dettagli funzionali e metriche.
\end{itemize}
\end{frame}

\begin{frame}{Metriche}
\textbf{Funzionalità Raggiunte}:
\begin{itemize}
    \item Rilevazioni near real-time nei 5 use case.
    \item Dettagli funzionali e metriche.
\end{itemize}
\end{frame}

\begin{frame}{Limitazioni}
\begin{itemize}
    \item Discussione delle limitazioni dell'approccio.
    \item L'azienda ha deciso di passare a una soluzione edge su mio consiglio, estensione di questo progetto.
    \item Consapevolezza delle differenze rispetto ai lavori accademici più complessi.
\end{itemize}
\end{frame}

\section{Conclusioni}

\begin{frame}{Conclusioni e Sviluppi futuri}
\begin{itemize}
    \item Riepilogo dei punti chiave.
    \item Potenziali miglioramenti ed estensioni future.
\end{itemize}
\end{frame}

\section*{Domande}

\backmatter
\end{document}
